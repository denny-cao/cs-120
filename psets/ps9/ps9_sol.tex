\documentclass[11pt]{scrartcl}
\usepackage[sexy]{../../../evan}
\usepackage{graphicx}
\usepackage{float}
\usepackage[noend,linesnumbered,boxed]{algorithm2e}
\usepackage{algpseudocode}

%%% Coloring the comment as blue
\newcommand\mycommfont[1]{\footnotesize\ttfamily\textcolor{blue}{#1}}
\SetCommentSty{mycommfont}

\SetKwInput{KwInput}{Input}                % Set the Input
\SetKwInput{KwOutput}{Output}              % set the Output

\definecolor{dg}{RGB}{2,101,15}
\newtheoremstyle{dotlessP}{}{}{}{}{\color{dg}\bfseries}{}{ }{}
\theoremstyle{dotlessP}
\newtheorem{property}[theorem]{Property}

\newtheoremstyle{dotlessN}{}{}{}{}{\color{teal}\bfseries}{}{ }{}
\theoremstyle{dotlessN}
\newtheorem{notation}[theorem]{Notation}
% Shortcuts
\DeclarePairedDelimiter\ceil{\lceil}{\rceil} % ceil function

\DeclarePairedDelimiter\paren{(}{)} % parenthesis

\newcommand{\df}{\displaystyle\frac} % displaystyle fraction
\newcommand{\qeq}{\overset{?}{=}} % questionable equality

\newcommand{\Mod}[1]{\;\mathrm{mod}\; #1} % modulo operator

\newcommand{\comp}{\circ} % composition

% Text Modifiers
\newcommand{\tbf}{\textbf}
\newcommand{\tit}{\textit}

% Sets
\DeclarePairedDelimiter\set{\{}{\}}
\newcommand{\unite}{\cup}
\newcommand{\inter}{\cap}

\newcommand{\reals}{\mathbb{R}} % real numbers: textbook is Z^+ and 0
\newcommand{\ints}{\mathbb{Z}}
\newcommand{\nats}{\mathbb{N}}
\newcommand{\complex}{\mathbb{C}}
\newcommand{\tots}{\mathbb{Q}}

\newcommand{\degree}{^\circ}

% Counting
\newcommand\perm[2][^n]{\prescript{#1\mkern-2.5mu}{}P_{#2}}
\newcommand\comb[2][^n]{\prescript{#1\mkern-0.5mu}{}C_{#2}}

% Relations
\newcommand{\rel}{\mathcal{R}} % relation

\setlength\parindent{0pt}

% Directed Graphs
\usetikzlibrary{arrows}
\usetikzlibrary{positioning,chains,fit,shapes,calc}

% Contradiction
\newcommand{\contradiction}{{\hbox{%
    \setbox0=\hbox{$\mkern-3mu\times\mkern-3mu$}%
    \setbox1=\hbox to0pt{\hss$\times$\hss}%
    \copy0\raisebox{0.5\wd0}{\copy1}\raisebox{-0.5\wd0}{\box1}\box0
}}}

% CS 
% NP
\newcommand{\np}{\texttt{NP}_\texttt{search}}
\newcommand{\p}{\texttt{P}_\texttt{search}}
\newcommand{\nph}{\texttt{NP}_\texttt{search}\text{-hard}}
\newcommand{\npc}{\texttt{NP}_\texttt{search}\text{-complete}}
\newcommand{\EXP}{\texttt{EXP}_\texttt{search}}
\newcommand{\xxhash}[2]{\rotatebox[origin=c]{#2}{$#1\parallel$}}

\title{CS 120: Intro to Algorithms and their Limitations}
\subtitle{PSet 6}
\author{Denny Cao}
\date{\today}
%++++++++++++++++++++++++++++++++++++++++
% title stuff
\usepackage{titling}
\renewcommand\maketitlehooka{\null\mbox{}\vfill}
\renewcommand\maketitlehookd{\vfill\null}
\makeatletter
\renewcommand{\maketitle}{\bgroup\setlength{\parindent}{0pt}
	\begin{flushleft}
		\large\textbf{\@title} \\ \vskip 0.2cm
		\begingroup
		\fontsize{14pt}{12pt}\selectfont
		\title
		\\
		Problem Set 9
		\endgroup \vskip 0.3cm
		Due: December 6, 2023 11:59pm \hfill\rlap{}\textbf{Denny Cao} \\ \vskip 0.1cm
		\hrulefill
	\end{flushleft}\egroup
}
\makeatother

\makeatletter
\newenvironment{algoproc}[1][]
  {\renewcommand{\algorithmcfname}{Procedure}%
   \begin{algorithm}[#1]
   \long\def\@caption##1[##2]##3{%
     \par
     \begingroup\@parboxrestore
     \if@minipage\@setminipage\fi
     \normalsize \@makecaption{\AlCapSty{\AlCapFnt\algorithmcfname}}{\ignorespaces ##3}%
     \par\endgroup
   }}
  {\end{algorithm}}
\makeatother

\renewcommand{\theques}{\thesection.\alph{ques}} % Change subtheo counter for alpha output
\declaretheorem[style=basehead,name=Answer,sibling=theorem]{ans}
\renewcommand{\theans}{\thesection.\alph{ans}}
\begin{document}
\maketitle
\pagestyle{plain}
\textbf{Collaborators:}

\textbf{No. of late days used on previous psets:} 6

\textbf{No. of late days used after including this pset:} 7
\section{$\p \not\subseteq \np$}
\begin{claim*}
	$\Pi \in \p$.
\end{claim*}
\begin{proof}
	[Proof of claim]
	To show that $\Pi$ is in $\p$, we show that there exists a Word-RAM program $P$ such that $f(P)$ can be found in polynomial time. Let $P$ be the Word-RAM program that returns whether or not an input $x \in \nats$ is at least 1:
\begin{algorithm}[h]                
        \SetKwInOut{Output}{Output~}
        \SetKwInOut{Input}{Input~}
		\SetKwInOut{Variables}{Variables~}
		\Input{~An input $x$ occupying memory location $M[0]$}
		\Output{~A value $y \in \set*{\texttt{0},\texttt{1}}$ in memory location $M[0]$}
		\Variables{~\texttt{input\_len}, \texttt{output\_len}, \texttt{output\_ptr}, \texttt{zero}, \texttt{word\_len}, \texttt{mem\_size}}
		\texttt{zero} = 0\texttt{;} \\
		\texttt{output\_len} = \texttt{input\_len} + \texttt{zero}\texttt{;} \\
		\texttt{output\_ptr} = 0\texttt{;}\\
		\textbf{if} $M[\texttt{output\_ptr}] >= 1$ \textbf{GOTO} \ref{6} \texttt{;} \\
		\textbf{if} $\texttt{zero} == 0$ \textbf{GOTO} \ref{7} \texttt{;}\\
		$M[\texttt{output\_ptr}] = 1$ \label{6} \texttt{;} \\
		\textbf{HALT}\label{7} \texttt{;}
		\caption{Word-RAM Program $P$}
\end{algorithm}

We verify that $f(P)$ can be found in polynomial time. We break inputs $\varepsilon \in \nats$ for $P$ into two cases:
\begin{enumerate}[1.]
	\item $\varepsilon \geq 1$: The total operations performed in this case is 6 regardless of the amount of bits to represent $\varepsilon$ in the Word-RAM program, and thus takes time $O(1)$.
	\item $\varepsilon= 0$: The total operations performed in this case is also 6, and thus takes time $O(1)$.
\end{enumerate}
As in both cases, $P$ halts in $O(1)$ and $P$ halts for all $\varepsilon \in \nats$, it follows that, for all inputs $\varepsilon$ for $P$, $f(P) = \set*{\texttt{0}, \texttt{1}}$ and can be found in $O(1)$ time which is polynomial time, and the proof is complete.
\end{proof}
\begin{claim*}
	$\Pi \not\in \np$. 
\end{claim*}
\begin{proof}
	[Proof of claim]
	To show that $\Pi \not\in \np$, we show that there exists a Word-RAM program $P$ such that a solution $x$ cannot be verified in polynomial time. Let $P$ be the Word-RAM program as follows:

\begin{algorithm}[H]                
        \SetKwInOut{Input}{Input~}
		\SetKwInOut{Variables}{Variables~}
		\Input{~An input $x$ occupying memory location $M[0]$}
		\Variables{~\texttt{input\_len}, \texttt{output\_len}, \texttt{output\_ptr}, \texttt{zero}, \texttt{word\_len}, \texttt{mem\_size}}
		\texttt{zero} = 0\texttt{;} \\
		\texttt{output\_len} = \texttt{input\_len} + \texttt{zero}\texttt{;} \\
		\texttt{output\_ptr} = 0\texttt{;}\\
		\textbf{if} $\texttt{zero} == 0$ \textbf{GOTO} \ref{4} \label{4} \texttt{;}\\
		\textbf{HALT}\texttt{;}
		\caption{Word-RAM Program $P$}
\end{algorithm}
Consider the case when the verifier $V$ must verify if a solution $x = \texttt{0}$ is correct for $P$. This will only happen if $P$ does not halt. However, if $P$ does not halt, then the verifier $V$ would run indefinitely, and thus $V$ would not run efficiently in polynomial time, and the proof is complete.
\end{proof}
\section{Undecidability of Arithmetic Overflows}
\begin{enumerate}[(a)]
	\item 
		\begin{proof}
			Let $P'$ be a Word-RAM equivalent program for a RAM program $P$. We form a new program $P''$ where integer overflows cannot occur by an algorithm $\varphi = \texttt{ConvertNoOverflowP}$ which takes $P'$ as input. We describe $\varphi$:
\begin{algorithm}                
	\caption{\texttt{ConvertNoOverflowP}}
  \SetKwInOut{Input}{inputs}
  \SetKwInOut{Output}{output}
  \SetKwProg{ConvertNoOverflowP}{ConvertNoOverflowP}{}{}

  \ConvertNoOverflowP{$(P'(V', (C'_0, \dots, C'_{\ell-1})))$}{
	 $C'' = (\ )$;\\
 	 \ForEach{$i \in [\ell]$}{%
		 \If{$C'_i$ in the form $\texttt{\upshape var}_i = \texttt{\upshape var}_j + \texttt{\upshape var}_k$ or $\texttt{\upshape var}_i = \texttt{\upshape var}_j \times \texttt{\upshape var}_k$}{%
			 $\texttt{start}_j = \texttt{len($C''$)} + 3$;

			 $\texttt{start}_k = \texttt{len($C''$)} + 5$;

			 $\texttt{end}_j = \texttt{len($C''$)} + 15$;

			 $\texttt{end}_k = \texttt{len($C''$)} + 13$;
				
			 $\texttt{end}_S = \texttt{len($C''$)} + 10$;

			 $C''\texttt{.append($\texttt{counter}_j = \texttt{0}$)}$;

			 $C''\texttt{.append($\texttt{counter}_S = \texttt{0}$)}$;

			 $C''\texttt{.append($\texttt{IF } \texttt{counter}_j == \texttt{var}_j \texttt{ GOTO } \texttt{end}_j$)}$;

			 $C''\texttt{.append($\texttt{counter}_k = \texttt{0}$)}$;

			 $C''\texttt{.append($\texttt{IF } \texttt{counter}_k == \texttt{var}_k \texttt{ GOTO } \texttt{end}_k$)}$;

			 $C''\texttt{.append($\texttt{IF } \texttt{counter}_S == S \texttt{ GOTO } \texttt{end}_S$)}$;
			 
			 $C''\texttt{.append($\texttt{counter}_S = \texttt{counter}_S + \texttt{one}$)}$;

			$C''\texttt{.append($\texttt{counter}_k = \texttt{counter}_k + \texttt{one}$)}$;

	 $C''\texttt{.append($\texttt{IF } \texttt{zero} == \texttt{0} \texttt{ GOTO } \texttt{start}_k$)}$;

			 $C''\texttt{.append(MALLOC())}$;


			 $C''\texttt{.append($\texttt{counter}_k = \texttt{counter}_k + \texttt{one}$)}$;

			 $C''\texttt{.append($\texttt{IF } \texttt{zero} == \texttt{0} \texttt{ GOTO } \texttt{start}_k$)}$;

			 $C''\texttt{.append($\texttt{counter}_j = \texttt{counter}_j + \texttt{one}$)}$;

			 $C''\texttt{.append($\texttt{IF } \texttt{zero} == \texttt{0} \texttt{ GOTO } \texttt{start}_j$)}$;
    }
	$C''\texttt{.append($C'_i$)}$;
}
	\KwRet{$P(V+3,C'')$};
}\end{algorithm}

For $P'$, integer overflow occurs when an operation $\texttt{var}_j \texttt{ op } \texttt{var}_k$ results in a value greater than or equal to $2^w$, where $w = \texttt{word\_len}$.
	We bound the greatest value of $\texttt{var}_j \texttt{ op } \texttt{var}_k$ by the maximum value of  $\texttt{var}_j \times \texttt{var}_k$, as multiplication will result in the largest value. We can thus bound the number of bits needed to represent an operation without integer overflow with
			\[
				\ceil*{\log_2(\texttt{var}_j \texttt{ op } \texttt{var}_k)} \leq \ceil*{\log_2(\texttt{var}_j \times \texttt{var}_k)}
			\] 
		We can avoid integer overflow by calling \texttt{MALLOC} $m$ times if $\texttt{var}_j \texttt{ op } \texttt{var}_k \geq 2^w$. The amount of times we call \texttt{MALLOC} will be 
					\[
	m =						\texttt{var}_j \times \texttt{var}_k - S
					\] 
					As $w = \floor{\log \max \set*{S, x[0], \dots, x[n-1]}}$, and thus  $w \geq \log S$, where $S$ is the length of memory, if we call \texttt{MALLOC} $\texttt{var}_j \times \texttt{var}_k - S$ times prior to $\texttt{var}_j \texttt{ op } \texttt{var}_k$, we will obtain a new value for $S$, $S'$, such that $S' \geq \texttt{var}_j \texttt{ op } \texttt{var}_k$. This is because, with each \texttt{MALLOC} call, we increment $S$ by 1.
					
From Theorem 7.1 in Lecture 7, for every RAM program $P$, there exists a Word-RAM Program $P'$ such that $P'$ halts on $x$ if and only if $P$ halts on $x$, and if they halt, then $P'(x) = P(x)$. It follows then that, if we show that, for every Word-RAM Program $P'$ there exists a Word-RAM Program $P''$ such that $P'$ halts on $x$ if and only if $P'$ halts on $x$, and if they halt, then $P''(x) = P'(x)$, by transitivity of biconditional statements, our proof is complete.
		\end{proof}
	\item 
		\begin{proof}
			It suffices to show that \texttt{HaltOnEmpty} reduces to \texttt{ArithmeticOverflow}, or that there is an algorithm $A$ that solves \texttt{HaltOnEmpty} given an oracle for \texttt{ArithmeticOverflow}. We describe the reduction below:
\begin{algorithm}                
	\caption{Reduction from \texttt{HaltOnEmpty} to \texttt{ArithmeticOverflow}}
  \SetKwInOut{Input}{Input~}
  \SetKwInOut{Output}{Output~}
  $A(P)$:

  \Input{~A RAM program $P$}
  \Output{~\texttt{yes} if $P$ halts on $\varepsilon$, \texttt{no} otherwise}
  Construct from $P$ a RAM program $Q_P$ such that $Q_P$ overflows on $\varepsilon$ if and only if $P$ halts on $\varepsilon$;

	Run the \texttt{ArithmeticOverflow} oracle on $Q_P$ and return its result;
\end{algorithm}

We construct $Q_P$ as follows:

\begin{algorithm}[H]
  $Q_P(P)$:

  $P' = \texttt{ConvertNoOverflowP($P$)}$;
  
  $C'' = (\ )$;

	$\texttt{end} = \texttt{len($P'.C$)}$;

  $C''\texttt{.append($\texttt{x} = 2$)}$;

	  $C''\texttt{.append($\texttt{counter} = 0$)}$;

	  $C''\texttt{.append($\texttt{IF } \texttt{counter} == \texttt{word\_len} \texttt{ GOTO } \texttt{end} + 7$)}$;

	  $C''\texttt{.append($\texttt{x} = \texttt{x} \times 2$)}$;

	  $C''\texttt{.append($\texttt{counter} = \texttt{counter} + 1$)}$;

	  $C''\texttt{.append($\texttt{IF } \texttt{zero} == 0 \texttt{ GOTO } \texttt{end} + 3$)}$;

	  $C''\texttt{.append(HALT)}$;
		
	  $Q_P = P'$;

	  \ForEach{$i \in [7]$}{
		  $Q_P.C$\texttt{.append($C''_i$)};
	  }
	  \KwRet{$Q_P$;}
	  \caption{The RAM Program $Q_P$ constructed from $P$}
\end{algorithm}

\begin{claim*}
	$Q_P$ overflows on $\varepsilon$ if and only if  $P$ halts on $\varepsilon$.
\end{claim*}
\begin{subproof}
	[Proof of claim]	
	We construct $Q_P$ by first creating an equivalent Word-RAM program $P'$ for $P$ such that there will be no integer overflows for operations by calling $\texttt{ConvertNoOverflowP($P$)}$ from Question 2.a. As $P'$ can still overflow by setting a variable to a value larger than $2^w$, we set a variable $x = 2^{w+1}$ if $P'$ halts. As $P'$ by construction does not result in overflows otherwise, by adding $C''$, we ensure that it halts if and only if it overflows
\end{subproof}
The claim implies that plugging this construction of $Q_P$ into Algorithm 4 gives a correct reduction from \texttt{HaltOnEmpty} to \texttt{ArithmeticOverflow}, and thus completes the proof that \texttt{ArithmeticOverflow} is unsolvable.
		\end{proof}
\end{enumerate}
\end{document}
