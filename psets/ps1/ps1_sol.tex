\documentclass[11pt]{scrartcl}
\usepackage[sexy]{../../../evan}
\usepackage{graphicx}

\definecolor{dg}{RGB}{2,101,15}
\newtheoremstyle{dotlessP}{}{}{}{}{\color{dg}\bfseries}{}{ }{}
\theoremstyle{dotlessP}
\newtheorem{property}[theorem]{Property}

\newtheoremstyle{dotlessN}{}{}{}{}{\color{teal}\bfseries}{}{ }{}
\theoremstyle{dotlessN}
\newtheorem{notation}[theorem]{Notation}
% Shortcuts
\DeclarePairedDelimiter\ceil{\lceil}{\rceil} % ceil function

\DeclarePairedDelimiter\paren{(}{)} % parenthesis

\newcommand{\df}{\displaystyle\frac} % displaystyle fraction
\newcommand{\qeq}{\overset{?}{=}} % questionable equality

\newcommand{\Mod}[1]{\;\mathrm{mod}\; #1} % modulo operator

\newcommand{\comp}{\circ} % composition

% Text Modifiers
\newcommand{\tbf}{\textbf}
\newcommand{\tit}{\textit}

% Sets
\DeclarePairedDelimiter\set{\{}{\}}
\newcommand{\unite}{\cup}
\newcommand{\inter}{\cap}

\newcommand{\reals}{\mathbb{R}} % real numbers: textbook is Z^+ and 0
\newcommand{\ints}{\mathbb{Z}}
\newcommand{\nats}{\mathbb{N}}
\newcommand{\complex}{\mathbb{C}}
\newcommand{\tots}{\mathbb{Q}}

\newcommand{\degree}{^\circ}

% Counting
\newcommand\perm[2][^n]{\prescript{#1\mkern-2.5mu}{}P_{#2}}
\newcommand\comb[2][^n]{\prescript{#1\mkern-0.5mu}{}C_{#2}}

% Relations
\newcommand{\rel}{\mathcal{R}} % relation

\setlength\parindent{0pt}

% Directed Graphs
\usetikzlibrary{arrows}
\tikzset{vertex/.style = {shape=circle,draw,minimum size=2em}}
\tikzset{svertex/.style = {shape=circle,draw,minimum size=.05em,font=\tiny}}
\tikzset{edge/.style = {->,> = latex'}}
\tikzset{dedge/.style = {-> = latex'}}
\tikzset{dot/.style={inner sep=1.5pt,circle,draw,fill}}


% Contradiction
\newcommand{\contradiction}{{\hbox{%
    \setbox0=\hbox{$\mkern-3mu\times\mkern-3mu$}%
    \setbox1=\hbox to0pt{\hss$\times$\hss}%
    \copy0\raisebox{0.5\wd0}{\copy1}\raisebox{-0.5\wd0}{\box1}\box0
}}}

\newcommand{\xxhash}[2]{\rotatebox[origin=c]{#2}{$#1\parallel$}}

\title{CS 120: Intro to Algorithms and their Limitations}
\subtitle{PSet 1}
\author{Denny Cao}
\date{\today}
%++++++++++++++++++++++++++++++++++++++++
% title stuff
\usepackage{titling}
\renewcommand\maketitlehooka{\null\mbox{}\vfill}
\renewcommand\maketitlehookd{\vfill\null}
\makeatletter
\renewcommand{\maketitle}{\bgroup\setlength{\parindent}{0pt}
	\begin{flushleft}
		\large\textbf{\@title} \\ \vskip 0.2cm
		\begingroup
		\fontsize{14pt}{12pt}\selectfont
		\title
		\\
		Problem Set 1
		\endgroup \vskip 0.3cm
		Due: September 20, 2023 11:59pm \hfill\rlap{}\textbf{Denny Cao} \\ \vskip 0.1cm
		\hrulefill
	\end{flushleft}\egroup
}
\makeatother

\renewcommand{\theques}{\thesection.\alph{ques}} % Change subtheo counter for alpha output
\declaretheorem[style=basehead,name=Answer,sibling=theorem]{ans}
\renewcommand{\theans}{\thesection.\alph{ans}}
\begin{document}
\maketitle
\pagestyle{plain}
\section{Asymptotic Notation}
\begin{ans} \
	\begin{table}[h!]
        \centering
        \bgroup
        \def\arraystretch{1.3}
        \begin{tabular}{||c | c || c | c | c | c | c ||}
         \hline
         $f$ & $g$ & $O$ & $o$ & $\Omega$ & $\omega$ & $\Theta$ \\
         \hline\hline
         $e^{n^2}$ & $e^{2n^2}$ & T & T & F & F & F \\ \hline
         $n^3$ & $n^{3/n}$ & F & F & T & T & F \\ \hline
		 $n^{2+(-1)^n}$ & $\binom{n}{2}$ & F & F & F & F & F \\ \hline
         $(\log {n})^{120}\sqrt{n}$ & $n$ & T & T & F & F & F \\ \hline
         $\log(e^{n^2})$ & $\log(e^{2n^2})$ & T & F & T & F & T \\ \hline
        \end{tabular}
        \egroup
    \end{table}
\end{ans}
\begin{ans} \

By definition of Big $\Theta$, we know that $f(n) = \Theta(a^n) \to f(n) = O(a^n) \land f(n) = \Omega(a^n)$. By definition of Big $O$ and Big $\Omega$,  $f(n) = O(a^n) \to \exists c_1 > 0 \mid f(n) \leq c_1a^n$ for all $n$ greater than some  $k$ and $f(n) = \Omega(a^n) \to \exists c_2 > 0 \mid f(n) \geq c_2a^n$ for all $n$ greater than some $k$. We combine into the following inequality: 
			\[
			c_1a^n \leq f(n) \leq c_2 a^n
			\] 
			Similarly, $g(n) = \Theta (n^b) \to g(n) = O(n^b) \land g(n) = \Omega(n^b)$. Thus, $g(n) = O(n^b) \to \exists c_3 > 0 \mid g(n) \leq c_3n^b$ for all $n$ greater than some $k$ and $g(n) = \Omega(n^b) \to \exists c_4 > 0 \mid g(n) \geq c_4 n^b$ for all $n$ greater than some $k$. We combine into the following inequality:
			\[
			c_3n^b \leq g(n) \leq c_4n^b 
			\] 
			We will use these statements to prove the following:
	\begin{itemize}
		\item \textbf{$f(g(n)) = \Theta(a^{(n^b)})$ is false.}
		\begin{proof}
			[Proof by counterexample]
			Let $f(n) = 2^n$ and  $g(n) = 10n^2$. We will show that $f(n) = \Theta(2^n)$. Let $c_1 = 1, c_2 =2$. Then, the following statement is true:
				\[
			2^n \leq f(n) \leq 2(2)^n
			\] 
			Thus, $f(n) = \Theta(2^n)$. We will show that  $g(n) = \Theta(x^2)$. Let $c_3 = 1, c_4 = 10$. Then, the following statement is true:
			\[
			n^2 \leq g(n) \leq 10n^2 
			\] 
			$f(g(n)) = 2^{10n^2} = {(2^{10})}^{n^2}$. Thus, $f(g(n)) \neq \Theta(2^{n^2})$, as there does not exist a $k_1$ and $k_2$ such that:
			\[
				f(g(n)) \leq k_2(2^{n^2})
			\] 
			Thus, $f(g(n)) \neq O(2^{n^2})$, which implies that  $f(g(n)) \neq \Theta(2^{n^2})$.
		\end{proof}
	\item  \textbf{$g(f(n)) = \Theta((a^n)^b)$ is true.}
		\begin{proof}
			Since for the composite $g(f(n))$, the least value of the input for  $g$ is the least value output for $f$ and the greatest value input for $g$ is the greatest value output for $f$ (Both functions are increasing), it is the case that:
			\begin{gather*}
				c_3 (c_1a^n)^b \leq g(f(n)) \leq c_4(c_1a^n)^b \\
				c_3c_1^b(a^n)^b \leq g(f(n)) \leq c_4c_1^b(a^n)^b
				\intertext{Let $r_1 = c_3c_1^b$ and $r_2 = c_4c_1^b$. Then:}
				r_1(a^n)^b \leq g(f(n)) \leq r_2(a^n)^b
			\end{gather*}
		As $\exists r_2 > 0 \mid g(f(n)) \leq r_2(a^n)^b$, by definition of Big $O$, $g(f(n)) = O((a^n)^b)$. As $\exists r_1 > 0 \mid g(f(n)) \geq r_1(a^n)^b$, by definition of Big $\Omega$, $g(f(n)) = \Omega((a^n)^b)$. Thus, as $g(f(n)) = O((a^n)^b) \land g(f(n)) = \Omega((a^n)^b)$, by definition of Big  $\Theta$, $g(f(n)) = \Theta((a^n)^b)$.
		\end{proof}
	\end{itemize}
\end{ans}
\section{Understanding Computational Problems and Mathematical Notation}
\begin{ans}
	If the input is $(11,10,4)$, then the output will be  $(1,1,0,0)$. \texttt{BC}'s output is a valid solution for $\Pi$ with input $(11,10,4)$, as the output satisfies $f(11,10,4)$, as $1 + 1(10) + 0(100) + 0(1000) = 11 = n$.
\end{ans}
\begin{ans}

\end{ans}
\begin{ans}
	When, for an $x = (n,b,k)$, $b < 2, b \in \nats$ ($b = 1$), or when $k$ is less than the amount of digits of $n$ expressed in base $b$.
\end{ans} 
\begin{ans}
	$|f(x)| = 1$, as there are not multiple representations of a number in base  $b$ ; every number has a unique representation in base $b$.
\end{ans}
\begin{ans}
	No, not every algorithm $A$ that solves $\Pi$ also solves $\Pi'$.
	\begin{proof}
		[Proof by counterexample]
		We will show that there exists an algorithm $A$ that solves $\Pi$ but does not solve $\Pi$. By definition of an algorithm, an algorithm $A$ solves a computational problem $\Pi$ if $\forall x \in \mathcal{I} (f(x) \neq \emptyset \to A(x) \in f(x))$. \texttt{BC} solves $\Pi$. We will show that \texttt{BC} does not solve $\Pi'$. Let $x = (11,1, 10)$. In line 2 of \texttt{BC}, \texttt{if b < 2 then return} $\perp$. As  $b = 1$, \texttt{BC} will return $\perp$, and thus $f(x) = \emptyset$. However, for the same input, $f'(x) = \emptyset \unite \set*{(0,1,\dots,9)} = \set*{0,1,\dots,9}$. As \texttt{BC} will return $\perp$ and not $\set*{0,1,\dots,9}$, there exists an algorithm that solves $\Pi$ that does not solve $\Pi'$.
	\end{proof}
\end{ans}
\section{Radix Sort}
\begin{ans} (Proving Correctness of Algorithms)\ 
	\begin{proof}
		[Proof by Induction] We will prove the correctness of Radix Sort by induction on the number of iterations of the inner loop in the algorithm. Let $P(n)$ be the statement that, at every step $n$, the subintegers of the elements within the input array $A$ become correctly ordered based on their corresponding digits, proceeding from the least significant digit to the most significant digit. 		\\
		
		\textit{Base Case}: $n=0$. The 0th digit of the elements in $A$ are sorted.
\\

		\textit{Inductive Hypothesis}: Assume that after $i$ iterations of the inner loop (sorting based on the $i$-th least significant digit), the subintegers of the elements in the array $A$ are correctly ordered based on their first $i$ digits (from right to left). We will show that $P(i) \to P(i + 1)$.
\\

\textit{Inductive Step}:  After the $i+ 1$th iteration, the elements will be sorted based on the $i$th digit from the right. As we use \texttt{CountingSort} as a subroutine, and \texttt{CountingSort} is stable, the order will be preserved. As $P(i)$ is true, the elements of $A$ are correctly sorted on their first $i$ digits (from right to left). Thus, after the $i+1$th iteration, if two numbers differ at the $i$th digit from the right, then they are placed in correct order and if they have the same value, then their position remains unchanged since \texttt{CountingSort} preserves order. Thus, the elements of $A$ will become correctly sorted based on their first $i+1$ digits (from right to left).
\\

Thus, with induction, we have shown that Radix Sort correctly solves the Sorting Problem.
	\end{proof}
\end{ans}
\begin{ans} (Analyzing Runtime) \
	\begin{proof}
		\texttt{CountingSort} takes $O(n + U)$ time when keys are drawn from a universe of size $U$. In \texttt{RadixSort}, \texttt{CountingSort} is used as a subroutine on a smaller universe, $b$. \texttt{CountingSort} is ran for each digit. Each key will have $\ceil*{\log U/\log b}$ digits in base $b$, as $\ceil*{\log U/\log b}$ will give the greatest power of $b$ that can be used to represent the largest value in the universe, $U$. The power will be the greatest amount of digits amongst all keys. As \texttt{CountingSort} is ran for each digit, there is a total running time of $(n+b)\ceil*{\log U/\log b}$. 
		\\

		The value of $b$ changes to minimize the expression depending on the values of $n$ and $U$. If $b = \min\{n,U\}$, then the expression will be: $O((n + n) (\log U / \log n)) = O(n\log U / \log n)$.
	\end{proof}
\end{ans}
\begin{ans} (Implementing Algorithms) In \texttt{ps1.py}.
\end{ans}
\begin{ans} (Experimentally Evaluating Algorithms) 
	The shapes of the resulting transition curves fit what asymptotic theory suggests. Around $2^5 = U$, there is a transition from \texttt{CountingSort} being more efficient to \texttt{MergeSort} being more efficient; asymptotic theory suggests that, with small values of $U$, \texttt{CountingSort} is more efficient and with larger values, \texttt{MergeSort} will be more efficient. Around the point where $U = n^{O(1)}$, \texttt{RadixSort} is most efficient, as it achieves runtime  $O(n)$ at that point.
\end{ans}
\end{document}

