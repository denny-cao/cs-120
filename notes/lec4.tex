\documentclass[11pt]{scrartcl}
\usepackage[sexy]{../../evan}
\usepackage{graphicx}

\definecolor{dg}{RGB}{2,101,15}
\newtheoremstyle{dotlessP}{}{}{}{}{\color{dg}\bfseries}{}{ }{}
\theoremstyle{dotlessP}
\newtheorem{property}[theorem]{Property}

\newtheoremstyle{dotlessN}{}{}{}{}{\color{teal}\bfseries}{}{ }{}
\theoremstyle{dotlessN}
\newtheorem{notation}[theorem]{Notation}
% Shortcuts
\DeclarePairedDelimiter\ceil{\lceil}{\rceil} % ceil function

\DeclarePairedDelimiter\paren{(}{)} % parenthesis

\newcommand{\df}{\displaystyle\frac} % displaystyle fraction
\newcommand{\qeq}{\overset{?}{=}} % questionable equality

\newcommand{\Mod}[1]{\;\mathrm{mod}\; #1} % modulo operator

\newcommand{\comp}{\circ} % composition

% Text Modifiers
\newcommand{\tbf}{\textbf}
\newcommand{\tit}{\textit}

% Sets
\DeclarePairedDelimiter\set{\{}{\}}
\newcommand{\unite}{\cup}
\newcommand{\inter}{\cap}

\newcommand{\reals}{\mathbb{R}} % real numbers: textbook is Z^+ and 0
\newcommand{\ints}{\mathbb{Z}}
\newcommand{\nats}{\mathbb{N}}
\newcommand{\complex}{\mathbb{C}}
\newcommand{\tots}{\mathbb{Q}}

\newcommand{\degree}{^\circ}

% Counting
\newcommand\perm[2][^n]{\prescript{#1\mkern-2.5mu}{}P_{#2}}
\newcommand\comb[2][^n]{\prescript{#1\mkern-0.5mu}{}C_{#2}}

% Relations
\newcommand{\rel}{\mathcal{R}} % relation

% Computational Problem Shortcuts
\newcommand{\I}{\mathcal{I}}

% Empty Set

\setlength\parindent{0pt}

% Directed Graphs
\usetikzlibrary{arrows}
\tikzset{vertex/.style = {shape=circle,draw,minimum size=2em}}
\tikzset{svertex/.style = {shape=circle,draw,minimum size=.05em,font=\tiny}}
\tikzset{edge/.style = {->,> = latex'}}
\tikzset{dedge/.style = {-> = latex'}}
\tikzset{dot/.style={inner sep=1.5pt,circle,draw,fill}}

% Contradiction
\newcommand\contradiction{\mathbin{\mathpalette\xhash\relax}}
\newcommand{\xhash}[2]{\ooalign{%
Ê $#1\xxhash{#1}{-45}$\cr
Ê $#1\xxhash{#1}{45}$\cr
Ê }%
}
\newcommand{\xxhash}[2]{\rotatebox[origin=c]{#2}{$#1\parallel$}}

\title{CS 120: Intro to Algorithms and their Limitations}
\subtitle{PSet 0}
\author{Denny Cao}
\date{\today}
%++++++++++++++++++++++++++++++++++++++++
% title stuff
\usepackage{titling}
\renewcommand\maketitlehooka{\null\mbox{}\vfill}
\renewcommand\maketitlehookd{\vfill\null}
\makeatletter
\renewcommand{\maketitle}{\bgroup\setlength{\parindent}{0pt}
	\begin{flushleft}
		\large\textbf{\@title} \\ \vskip 0.2cm
		\begingroup
		\fontsize{14pt}{12pt}\selectfont
		\title
		\\
		\normalsize Lecture 5:  --- Tuesday September 19, 2023 
		\endgroup \vskip 0.3cm
		\normalsize Pset Due: September 20, 2023 \hfill\rlap{}\textbf{Denny Cao} \\ \vskip 0.1cm
		\hrulefill
	\end{flushleft}\egroup
}
\makeatother

\renewcommand{\theques}{\thesection.\alph{ques}} % Change subtheo counter for alpha output
\declaretheorem[style=basehead,name=Answer,sibling=theorem]{ans}
\renewcommand{\theans}{\thesection.\alph{ans}}
\begin{document}
\maketitle
\pagestyle{plain}
\section{Announcements}
\begin{itemize}
	\item SRE next Thursday
	\item Wednesday morning OH on Zoom
	\item Late Policy: 3 days max for each assignment. Late days  $\in \nats$
\end{itemize}
\section{Dynamic Predecessor Data Structure}
For a sorted array:
\begin{itemize}
	\item Insert$(k,v) = O(n)$
	\item Delete$(k) = O(n)$
	\item Search$(k) = O(\log n)$
	\item Next-smaller$(k) = O(\log n)$
\end{itemize}
\section{Binary Search Tree}
Our goal is to solve Dynamic Predecessors more easily.
\begin{definition}
	[Binary Search Tree (BST)]
	Data structure defined \textbf{recursively}. Base case: $\emptyset$ or has a root $R$ and every vertex has:
	\begin{itemize}
		\item Key $K$ 
		\item Value $V$ 
		\item Left and Right Children $V_\text{left}$ and  $V_\text{right}$, each a BST (Could be a $\emptyset$)	
		\item Satisfies the \textbf{BST Property}
	\end{itemize}
\end{definition}
% Format this later
\begin{property}
	[BST Property]	
	$\forall v:$ if $v$ has left child $v_l$ : 

	the keys of $v_l$ and all its descendants are $\leq k_v$.

	If $v$ has a right child, similar definition.
\end{property}
% Format this later
\begin{algorithm}
	\begin{verbatim}
		If $T = \emptyset$	
	\end{verbatim}
	[Insert]
	Insert$(T(k,v))$ :

	if $T = \emptyset$: Return new BST w/key $K$, value $V$, no children

	Let $v = \text{root}(T)$

	if  $k \leq K_V$:
	
	T.Left = Insert$(T_L(K,V))$
	
	else \texttt{T.right} = Insert$(T_R(K,v))$
	return  $T$
\end{algorithm}
\begin{definition}
	[Height $h$]
	Length of the longest path from $v$ to a leaf.
	\begin{itemize}
		\item \texttt{height($\emptyset$) = -1}
	\end{itemize}
\end{definition}
\begin{proof}
	[Proof that Insert Maintains BST Property]
By induction on height \texttt{h} ``\texttt{Insert}($T(K,V)$) maintains BST if \texttt{height(T)} $\leq \texttt{h}$''

	\textit{Base Case}: $h = -1$: No vertices  $\to$ no vertices to check.

	If true for $< h: Insert (T(K,V))$:  \texttt{T} has a root  \texttt{v}, so  $\texttt{v} = \texttt{root(T)}$ is well defined.

	If $k \leq k_v$:  $Insert(T_L(K,V))$:  $T_L$ has height $\leq h-1$
\end{proof}
\begin{itemize}
	\item $Insert(k,v) = O(h)$
	\item $Delete(k) = O(h)$
	\item $Search(k) = O(h)$ \\
	\item $Next-Smaller(k) = O(h)$
\end{itemize}
\begin{property}
	[AVL Property]
	\begin{itemize}
		\item Every vertex is augmented with height 
		\item Every pair of siblings have heights differing by $\leq 1$
	\end{itemize}
\end{property}
\begin{itemize}
	\item Maintain balance by rotations: Left rotation, right rotation,
\end{itemize}
\end{document}
