\documentclass[11pt]{scrartcl}
\usepackage[sexy]{../../evan}
\usepackage{graphicx}

\definecolor{dg}{RGB}{2,101,15}
\newtheoremstyle{dotlessP}{}{}{}{}{\color{dg}\bfseries}{}{ }{}
\theoremstyle{dotlessP}
\newtheorem{property}[theorem]{Property}

\newtheoremstyle{dotlessN}{}{}{}{}{\color{teal}\bfseries}{}{ }{}
\theoremstyle{dotlessN}
\newtheorem{notation}[theorem]{Notation}
% Shortcuts
\DeclarePairedDelimiter\ceil{\lceil}{\rceil} % ceil function

\DeclarePairedDelimiter\paren{(}{)} % parenthesis

\newcommand{\df}{\displaystyle\frac} % displaystyle fraction
\newcommand{\qeq}{\overset{?}{=}} % questionable equality

\newcommand{\Mod}[1]{\;\mathrm{mod}\; #1} % modulo operator

\newcommand{\comp}{\circ} % composition

% Text Modifiers
\newcommand{\tbf}{\textbf}
\newcommand{\tit}{\textit}

% Sets
\DeclarePairedDelimiter\set{\{}{\}}
\newcommand{\unite}{\cup}
\newcommand{\inter}{\cap}

\newcommand{\reals}{\mathbb{R}} % real numbers: textbook is Z^+ and 0
\newcommand{\ints}{\mathbb{Z}}
\newcommand{\nats}{\mathbb{N}}
\newcommand{\complex}{\mathbb{C}}
\newcommand{\tots}{\mathbb{Q}}

\newcommand{\degree}{^\circ}

% Counting
\newcommand\perm[2][^n]{\prescript{#1\mkern-2.5mu}{}P_{#2}}
\newcommand\comb[2][^n]{\prescript{#1\mkern-0.5mu}{}C_{#2}}

% Relations
\newcommand{\rel}{\mathcal{R}} % relation

% Computational Problem Shortcuts
\newcommand{\I}{\mathcal{I}}

% Empty Set

\setlength\parindent{0pt}

% Directed Graphs
\usetikzlibrary{arrows}
\tikzset{vertex/.style = {shape=circle,draw,minimum size=2em}}
\tikzset{svertex/.style = {shape=circle,draw,minimum size=.05em,font=\tiny}}
\tikzset{edge/.style = {->,> = latex'}}
\tikzset{dedge/.style = {-> = latex'}}
\tikzset{dot/.style={inner sep=1.5pt,circle,draw,fill}}

% Contradiction
\newcommand\contradiction{\mathbin{\mathpalette\xhash\relax}}
\newcommand{\xhash}[2]{\ooalign{%
Ê $#1\xxhash{#1}{-45}$\cr
Ê $#1\xxhash{#1}{45}$\cr
Ê }%
}
\newcommand{\xxhash}[2]{\rotatebox[origin=c]{#2}{$#1\parallel$}}

\title{CS 120: Intro to Algorithms and their Limitations}
\subtitle{PSet 0}
\author{Denny Cao}
\date{\today}
%++++++++++++++++++++++++++++++++++++++++
% title stuff
\usepackage{titling}
\renewcommand\maketitlehooka{\null\mbox{}\vfill}
\renewcommand\maketitlehookd{\vfill\null}
\makeatletter
\renewcommand{\maketitle}{\bgroup\setlength{\parindent}{0pt}
	\begin{flushleft}
		\large\textbf{\@title} \\ \vskip 0.2cm
		\begingroup
		\fontsize{14pt}{12pt}\selectfont
		\title
		\\
		\normalsize Lecture :  --- Tuesday November 7, 2023
		\endgroup \vskip 0.3cm
		\normalsize Pset Due: November 15, 2023 \hfill\rlap{}\textbf{Denny Cao} \\ \vskip 0.1cm
		\hrulefill
	\end{flushleft}\egroup
}
\makeatother

\renewcommand{\theques}{\thesection.\alph{ques}} % Change subtheo counter for alpha output
\declaretheorem[style=basehead,name=Answer,sibling=theorem]{ans}
\renewcommand{\theans}{\thesection.\alph{ans}}
\begin{document}
\maketitle
\pagestyle{plain}
\section{Church-Turing Thesis}
\begin{claim}
	[Extended Church-Turing Thesis]
	Every physically realizable \textit{deterministic sequential} model of computation is simulable by Word-RAM \textit{with at most polynomial slowdown}.
\end{claim}
\section{Problem-Independent Size Measure}
\begin{itemize}
	\item Measure input size in bits to ecode
\end{itemize}
\begin{example}
	Graph w/$|V| = n, |E| = m$
\end{example}
\begin{itemize}
	\item $(1+m)\log n = N$
\end{itemize}
\begin{example}
	3-SAT formula: $n$ variables and $m$ clauses.
\end{example}
\begin{itemize}
\item For each $m$ clause, need what each of the variables are and if they are negated. $m\cdot 3 \cdot (1 + \log n)$, must specify each of the iterals, for each iteral, need 1 bit for if negated, and a variable or their negation.
\end{itemize}
\begin{example}
	Array of size $n$ from a universe $[U]$.
\end{example}
\begin{itemize}
	\item $n \log U = N$
	\item Thus, \texttt{RadixSort} is time $O(N + 2^N)$ (find explanation for why $2^N$, but it had something to do with how we can find the upperbound of $\log N$ or something)
\end{itemize}
\begin{lemma}
	[Reductions]
	If $\Pi \leq_{T,f} \Gamma$, then:
	\begin{itemize}
		\item $\exists$ alg solving $\Gamma$ in time $g(n) \to \exists$ alg solving $\Pi$ in time $O(g(f(n) + T(n)))$.
		\item Contrapositive: $\not\exists$ alg solving $\Pi$ in time $O(g(f(n)) + T(n))\to \not\exists$ alg solving $\Gamma$ in time $O(g(n))$
	\end{itemize}
\end{lemma}
\begin{itemize}
	\item If only polynomial runtimes, easier lemma
\end{itemize}
\begin{lemma}
	[Simpler Lemma w/Polynomial Time Complexity]
	If $\Pi \leq \Gamma$, then:
	\begin{itemize}
		\item $\exists$ alg solving $\Gamma$ in time something polynomial $n^{O(1)} \to \exists$ alg solving $\Pi$ in polynomial time $n^{O(1)}$.
		\item Contrapositive: $\not\exists$ alg solving $\Pi$ in $n^{O(1)} \to \not\exists$ alg solving $\Gamma$ in $n^{O(1)}$.
	\end{itemize}
\end{lemma}
\begin{proof}
	Reduction $R$ from $\Pi$ to $\Gamma$ which runs in time $O(n^c)$ for some $c$. The reduction gives an algorithm by oracle replacement. Suppose $\Gamma \in \text{TIME}(n^d)$. We replace oracle calls with runs of  $O(n^d)$, giving an algorithm that solves $\Pi$ with runtime:
	\begin{itemize}
		\item Reduction: $O(n^c)$
		\item Number of oracle calls: $O(n^c)$. We cannot now just substitute $n^c$ for  $n$ for the oracle replacement, as the size may be bigger
		\item Size of memory starts at $\leq n$, grows at $\leq 1$ per step, thus \texttt{mem\_size} $\leq n^c \to $ all oracle calls are of size $\leq n^c$.
		\item Now, we can find upperbound of each oracle call runtime: maximum size of $n$: $O((n^{c})^d)$, thus total runtime of oracle calls is $O(n^{cd + c})$
	\end{itemize}
\end{proof}
\begin{remark}
	We made the proof easier; there is something to fill in for when we call MALLOC, as MALLOC does not just grow memory by 1---could be an entire word for example.
\end{remark}
\begin{remark*}
	Can 3-coloring be solved in polynomial time?
\end{remark*}
\begin{theorem}
	The problem \texttt{HALT-IN-$2^n$}, whose input is a Word-RAM program $P$, $n \in \nats$ and output $\texttt{True} \iff P$ runs in time $\leq 2^n$ on all inputs of size $n$, then the \textbf{problem cannot be solved in polynomial time}.
\end{theorem}
\begin{proof}
	CS 121 :D
\end{proof}
\begin{itemize}
	\item Running on all inputs takes time $2^n$, exponential time.
	\item If \texttt{HALT-IN-$2^n$} is $\Pi$, then a reduction $\Gamma$ cannot be solved in polynomial time
	\item Must come up with a reduction
\end{itemize}
\section{Complexity Classes}
\begin{definition}
	[Complexity Classes]
	Sets of problems defined by difficulty of solving.
\end{definition}
\begin{itemize}
	\item We define $\text{Time}_\text{Search}(f(n)) =$ the set of Problems s.t. $\exists$ Word-RAM model solving $\Pi$ in time $O(f(n))$.
\end{itemize}
\begin{definition}
	[Polynomial Search] $P_\text{Search} = \displaystyle\bigcup_{c \in \nats} \text{TIME}_\text{Search}(n^c)$.
\end{definition}
\begin{definition}
	[Exponential Search] $\text{EXP}_\text{Search} = \displaystyle\bigcup_{c \in \nats}\text{TIME}_\text{Search}(2^{n^c})$.
\end{definition}
\begin{remark}
	Any problem in $P_\text{Search}$ is in $\text{EXP}_\text{Search}$, as they would be bounded by it.
\end{remark}
\begin{definition}
	[Decision Problems]
	$P = \displaystyle\bigcup_c \text{Time}(n^c)$ and
	$\text{EXP} = \displaystyle\bigcup_c \text{Time}(2^{n^c})$
\end{definition}
\begin{itemize}
	\item Uses bit length $n$
\item Comparison between complexity classes using Theorem 2.7: EXP contains P, but wonder if EXP = P. We now know that \texttt{HALT-IN-$2^n$} $\in \text{EXP} \land \not\in P$, and thus $P \subsetneq E \times P$.
\end{itemize}
\end{document}
