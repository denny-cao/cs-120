\documentclass[11pt]{scrartcl}
\usepackage[sexy]{../../evan}
\usepackage{graphicx}

\definecolor{dg}{RGB}{2,101,15}
\newtheoremstyle{dotlessP}{}{}{}{}{\color{dg}\bfseries}{}{ }{}
\theoremstyle{dotlessP}
\newtheorem{property}[theorem]{Property}

\newtheoremstyle{dotlessN}{}{}{}{}{\color{teal}\bfseries}{}{ }{}
\theoremstyle{dotlessN}
\newtheorem{notation}[theorem]{Notation}
% Shortcuts
\DeclarePairedDelimiter\ceil{\lceil}{\rceil} % ceil function

\DeclarePairedDelimiter\paren{(}{)} % parenthesis

\newcommand{\df}{\displaystyle\frac} % displaystyle fraction
\newcommand{\qeq}{\overset{?}{=}} % questionable equality

\newcommand{\Mod}[1]{\;\mathrm{mod}\; #1} % modulo operator

\newcommand{\comp}{\circ} % composition

% Text Modifiers
\newcommand{\tbf}{\textbf}
\newcommand{\tit}{\textit}

% Sets
\DeclarePairedDelimiter\set{\{}{\}}
\newcommand{\unite}{\cup}
\newcommand{\inter}{\cap}

\newcommand{\reals}{\mathbb{R}} % real numbers: textbook is Z^+ and 0
\newcommand{\ints}{\mathbb{Z}}
\newcommand{\nats}{\mathbb{N}}
\newcommand{\complex}{\mathbb{C}}
\newcommand{\tots}{\mathbb{Q}}

\newcommand{\degree}{^\circ}

% Counting
\newcommand\perm[2][^n]{\prescript{#1\mkern-2.5mu}{}P_{#2}}
\newcommand\comb[2][^n]{\prescript{#1\mkern-0.5mu}{}C_{#2}}

% Relations
\newcommand{\rel}{\mathcal{R}} % relation

% Computational Problem Shortcuts
\newcommand{\I}{\mathcal{I}}

% Empty Set

\setlength\parindent{0pt}

% Directed Graphs
\usetikzlibrary{arrows}
\tikzset{vertex/.style = {shape=circle,draw,minimum size=2em}}
\tikzset{svertex/.style = {shape=circle,draw,minimum size=.05em,font=\tiny}}
\tikzset{edge/.style = {->,> = latex'}}
\tikzset{dedge/.style = {-> = latex'}}
\tikzset{dot/.style={inner sep=1.5pt,circle,draw,fill}}

% Contradiction
\newcommand\contradiction{\mathbin{\mathpalette\xhash\relax}}
\newcommand{\xhash}[2]{\ooalign{%
Ê $#1\xxhash{#1}{-45}$\cr
Ê $#1\xxhash{#1}{45}$\cr
Ê }%
}
\newcommand{\xxhash}[2]{\rotatebox[origin=c]{#2}{$#1\parallel$}}

\title{CS 120: Intro to Algorithms and their Limitations}
\subtitle{PSet 0}
\author{Denny Cao}
\date{\today}
%++++++++++++++++++++++++++++++++++++++++
% title stuff
\usepackage{titling}
\renewcommand\maketitlehooka{\null\mbox{}\vfill}
\renewcommand\maketitlehookd{\vfill\null}
\makeatletter
\renewcommand{\maketitle}{\bgroup\setlength{\parindent}{0pt}
	\begin{flushleft}
		\large\textbf{\@title} \\ \vskip 0.2cm
		\begingroup
		\fontsize{14pt}{12pt}\selectfont
		\title
		\\
		\normalsize Lecture 6:  --- Thursday September 21, 2023 
		\endgroup \vskip 0.3cm
		\normalsize Pset Due: September 27, 2023 \hfill\rlap{}\textbf{Denny Cao} \\ \vskip 0.1cm
		\hrulefill
	\end{flushleft}\egroup
}
\makeatother

\renewcommand{\theques}{\thesection.\alph{ques}} % Change subtheo counter for alpha output
\declaretheorem[style=basehead,name=Answer,sibling=theorem]{ans}
\renewcommand{\theans}{\thesection.\alph{ans}}
\begin{document}
\maketitle
\pagestyle{plain}
\section{RAM Model}
\subsection{Goals}
\begin{itemize}
	\item Unambiguous: Know what a step is 
	\item Expressivity: Capture what we think of as algorithm
	\item Mathematical Simplicity: Not too many operations
	\item Technical Relevances: One model applies to all languages
\end{itemize}
\begin{definition}
	[RAM Model]
	A program consists of:
	\begin{itemize}
		\item A set $V$ of variables
		\item Commands $C_0,C_1,\dots,C_n$ each:
			\begin{itemize}
				\item $\text{var}_i=c$, $\text{var}_i \in V, c \in nats$
				\item $\text{var}_0 = \text{var}_1 \text{ op } \text{var}_2$, $\text{var}_0, \text{var}_1, \text{var}_2 \in V$, $\text{op} \in \set*{+, -, \times, \div}$
				\item $\text{var}_0 = M[\text{var}_1]$
				\item  $M[\text{var}_0] = \text{var}_1$
				\item If  $\text{var}==0$, GO TO  $K$,  $K \in \set*{0_1, \dots l}$
			\end{itemize}
	\end{itemize}
\end{definition}
\begin{definition}
	[Computation w/RAM Program] \

	\textit{Initialize}: Encode input in memory $M[0], \dots, M[n-1]$ (input is size $n$) and $M[k]=0$ if $k \geq n$.
	\\

	\textit{Execution}: Execute $C_0, C_1, C_2, \dots$ except that GOTO commands change order.
	\\

		\renewcommand{\defeq}{\vcentcolon=}
	\textit{Output}: When the program reaches line $l$, output $\defeq M[\text{output\_ptr}], \dots, M[\text{output} + \text{output\_len}]$
	$P(x) = \perp$ if it halts, $\text{Time}_p(x) = $ \# commands executed
\end{definition}
\end{document}
