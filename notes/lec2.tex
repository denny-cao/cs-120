\documentclass[11pt]{scrartcl}
\usepackage[sexy]{../../evan}
\usepackage{graphicx}

\definecolor{dg}{RGB}{2,101,15}
\newtheoremstyle{dotlessP}{}{}{}{}{\color{dg}\bfseries}{}{ }{}
\theoremstyle{dotlessP}
\newtheorem{property}[theorem]{Property}

\newtheoremstyle{dotlessN}{}{}{}{}{\color{teal}\bfseries}{}{ }{}
\theoremstyle{dotlessN}
\newtheorem{notation}[theorem]{Notation}
% Shortcuts
\DeclarePairedDelimiter\ceil{\lceil}{\rceil} % ceil function

\DeclarePairedDelimiter\paren{(}{)} % parenthesis

\newcommand{\df}{\displaystyle\frac} % displaystyle fraction
\newcommand{\qeq}{\overset{?}{=}} % questionable equality

\newcommand{\Mod}[1]{\;\mathrm{mod}\; #1} % modulo operator

\newcommand{\comp}{\circ} % composition

% Text Modifiers
\newcommand{\tbf}{\textbf}
\newcommand{\tit}{\textit}

% Sets
\DeclarePairedDelimiter\set{\{}{\}}
\newcommand{\unite}{\cup}
\newcommand{\inter}{\cap}

\newcommand{\reals}{\mathbb{R}} % real numbers: textbook is Z^+ and 0
\newcommand{\ints}{\mathbb{Z}}
\newcommand{\nats}{\mathbb{N}}
\newcommand{\complex}{\mathbb{C}}
\newcommand{\tots}{\mathbb{Q}}

\newcommand{\degree}{^\circ}

% Counting
\newcommand\perm[2][^n]{\prescript{#1\mkern-2.5mu}{}P_{#2}}
\newcommand\comb[2][^n]{\prescript{#1\mkern-0.5mu}{}C_{#2}}

% Relations
\newcommand{\rel}{\mathcal{R}} % relation

% Computational Problem Shortcuts
\newcommand{\I}{\mathcal{I}}

% Empty Set

\setlength\parindent{0pt}

% Directed Graphs
\usetikzlibrary{arrows}
\tikzset{vertex/.style = {shape=circle,draw,minimum size=2em}}
\tikzset{svertex/.style = {shape=circle,draw,minimum size=.05em,font=\tiny}}
\tikzset{edge/.style = {->,> = latex'}}
\tikzset{dedge/.style = {-> = latex'}}
\tikzset{dot/.style={inner sep=1.5pt,circle,draw,fill}}

% Contradiction
\newcommand\contradiction{\mathbin{\mathpalette\xhash\relax}}
\newcommand{\xhash}[2]{\ooalign{%
Ê $#1\xxhash{#1}{-45}$\cr
Ê $#1\xxhash{#1}{45}$\cr
Ê }%
}
\newcommand{\xxhash}[2]{\rotatebox[origin=c]{#2}{$#1\parallel$}}

\title{CS 120: Intro to Algorithms and their Limitations}
\subtitle{PSet 0}
\author{Denny Cao}
\date{\today}
%++++++++++++++++++++++++++++++++++++++++
% title stuff
\usepackage{titling}
\renewcommand\maketitlehooka{\null\mbox{}\vfill}
\renewcommand\maketitlehookd{\vfill\null}
\makeatletter
\renewcommand{\maketitle}{\bgroup\setlength{\parindent}{0pt}
	\begin{flushleft}
		\large\textbf{\@title} \\ \vskip 0.2cm
		\begingroup
		\fontsize{14pt}{12pt}\selectfont
		\title
		\\
		\normalsize Lecture 3: Reduction --- Thursday, September 7, 2023
		\endgroup \vskip 0.3cm
		\normalsize Pset Due: September 13, 2023 \hfill\rlap{}\textbf{Denny Cao} \\ \vskip 0.1cm
		\hrulefill
	\end{flushleft}\egroup
}
\makeatother

\renewcommand{\theques}{\thesection.\alph{ques}} % Change subtheo counter for alpha output
\declaretheorem[style=basehead,name=Answer,sibling=theorem]{ans}
\renewcommand{\theans}{\thesection.\alph{ans}}
\begin{document}
\maketitle
\pagestyle{plain}
\section{Comparison-Based Sorting}
\begin{itemize}
	\item Exhaustive Search
	\item Insertion Sort 
	\item Merge Sort
\end{itemize}
\begin{proof}
	[Proof of Lower Bound]
	The input is: $((\sigma(0), \emptyset), \dots, (\sigma(n), \emptyset)$. 

The output is  $((0, \emptyset), \dots, (n, \emptyset))$.
	\\
	$2^T > n!$. Represents number of possible outcomes (permutations) will be at least $n!$. We take the log of both sides = $\Omega \paren*{\displaystyle \frac{n}{e}^n}$
\end{proof}
\begin{remark}
	Look over lecture notes... Don't understand
\end{remark}
\section{Reductions}
\subsection{Duplicate Search}
\begin{problem}
	[Duplicate Search Problem]	
	\
	Input: $a_0, \dots, a_{n-1}$ 

	Output: A duplicate number; such that $\exists i \neq j \mid a_i = a_j = a$
\end{problem}
\begin{algorithm}
	[$O(n^2)$: Naive Approach]
	For each $a_i$, loop through all  $a_j$, $i<j\leq n-1$. Return  $a_i$ if duplicate found else  $\perp$.
\end{algorithm}
\begin{algorithm}
	[$O(n\log n)$: Sorted Array Approach]	
	\
	\begin{enumerate}
	\item Form an array $(k_0, v_0), \dots, (k_{n-1}, v_{n-1})$ with $k_i = a_i, v_i = \emptyset$: Output  $S$
	\item Sort
	\item For $0 \leq i < n - 1$, check if  $S[i] = S[i+1]$
	\end{enumerate}
\end{algorithm}
\begin{proof}
	[Proof of Correctness]
	Check lecture notes.
\end{proof}
\begin{remark}
	By convention, sorting problems are for key-value pairs. $v_i = \emptyset$ because we are ignoring the value and just sorting keys.
\end{remark}
We will formalize reduction:
\begin{definition}
	[Formalism of Reduction]
	\
	Input $\Pi = (\mathcal{I}, \mathcal{O}, f)$ 

	\textit{Oracle}  $\Gamma = (J, \varphi, g)$. 
\end{definition}
\begin{definition}
	[Oracle]
	An \textit{oracle} is a function which given $x \in \mathcal{J}$, outputs on element of $g(x)$.
\end{definition}
\begin{definition}
	[Reduction]
A \textit{reduction} is an algorithm that solves $\Pi$ using an oracle or $\Gamma$ as a subroutine.
\end{definition}
\begin{remark}
	$\Pi \leq \Gamma$	
\end{remark}
\begin{remark}
	For duplication search algorithm 2.3, oracle is the sorting algorithm. We don't have to prove the oracle again. Do not need to be specific about oracle; can just say ``sort''.
\end{remark}
\end{document}
