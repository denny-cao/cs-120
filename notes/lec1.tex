\documentclass[11pt]{scrartcl}
\usepackage[sexy]{../../evan}
\usepackage{graphicx}

\definecolor{dg}{RGB}{2,101,15}
\newtheoremstyle{dotlessP}{}{}{}{}{\color{dg}\bfseries}{}{ }{}
\theoremstyle{dotlessP}
\newtheorem{property}[theorem]{Property}

\newtheoremstyle{dotlessN}{}{}{}{}{\color{teal}\bfseries}{}{ }{}
\theoremstyle{dotlessN}
\newtheorem{notation}[theorem]{Notation}
% Shortcuts
\DeclarePairedDelimiter\ceil{\lceil}{\rceil} % ceil function

\DeclarePairedDelimiter\paren{(}{)} % parenthesis

\newcommand{\df}{\displaystyle\frac} % displaystyle fraction
\newcommand{\qeq}{\overset{?}{=}} % questionable equality

\newcommand{\Mod}[1]{\;\mathrm{mod}\; #1} % modulo operator

\newcommand{\comp}{\circ} % composition

% Text Modifiers
\newcommand{\tbf}{\textbf}
\newcommand{\tit}{\textit}

% Sets
\DeclarePairedDelimiter\set{\{}{\}}
\newcommand{\unite}{\cup}
\newcommand{\inter}{\cap}

\newcommand{\reals}{\mathbb{R}} % real numbers: textbook is Z^+ and 0
\newcommand{\ints}{\mathbb{Z}}
\newcommand{\nats}{\mathbb{N}}
\newcommand{\complex}{\mathbb{C}}
\newcommand{\tots}{\mathbb{Q}}

\newcommand{\degree}{^\circ}

% Counting
\newcommand\perm[2][^n]{\prescript{#1\mkern-2.5mu}{}P_{#2}}
\newcommand\comb[2][^n]{\prescript{#1\mkern-0.5mu}{}C_{#2}}

% Relations
\newcommand{\rel}{\mathcal{R}} % relation

% Computational Problem Shortcuts
\newcommand{\I}{\mathcal{I}}

% Empty Set

\setlength\parindent{0pt}

% Directed Graphs
\usetikzlibrary{arrows}
\tikzset{vertex/.style = {shape=circle,draw,minimum size=2em}}
\tikzset{svertex/.style = {shape=circle,draw,minimum size=.05em,font=\tiny}}
\tikzset{edge/.style = {->,> = latex'}}
\tikzset{dedge/.style = {-> = latex'}}
\tikzset{dot/.style={inner sep=1.5pt,circle,draw,fill}}

% Contradiction
\newcommand\contradiction{\mathbin{\mathpalette\xhash\relax}}
\newcommand{\xhash}[2]{\ooalign{%
Ê $#1\xxhash{#1}{-45}$\cr
Ê $#1\xxhash{#1}{45}$\cr
Ê }%
}
\newcommand{\xxhash}[2]{\rotatebox[origin=c]{#2}{$#1\parallel$}}

\title{CS 120: Intro to Algorithms and their Limitations}
\subtitle{PSet 0}
\author{Denny Cao}
\date{\today}
%++++++++++++++++++++++++++++++++++++++++
% title stuff
\usepackage{titling}
\renewcommand\maketitlehooka{\null\mbox{}\vfill}
\renewcommand\maketitlehookd{\vfill\null}
\makeatletter
\renewcommand{\maketitle}{\bgroup\setlength{\parindent}{0pt}
	\begin{flushleft}
		\large\textbf{\@title} \\ \vskip 0.2cm
		\begingroup
		\fontsize{14pt}{12pt}\selectfont
		\title
		\\
		\normalsize Lecture 2: Measuring Efficiency Thursday, September 7, 2023
		\endgroup \vskip 0.3cm
		\normalsize Pset Due: September 13, 2023 \hfill\rlap{}\textbf{Denny Cao} \\ \vskip 0.1cm
		\hrulefill
	\end{flushleft}\egroup
}
\makeatother

\renewcommand{\theques}{\thesection.\alph{ques}} % Change subtheo counter for alpha output
\declaretheorem[style=basehead,name=Answer,sibling=theorem]{ans}
\renewcommand{\theans}{\thesection.\alph{ans}}
\begin{document}
\maketitle
\pagestyle{plain}
\section{Announcements}
\begin{itemize}
	\item SRE starts Tuesday
	\item Participation surveys (see course website)
\end{itemize}
\section{Review}
\subsection{Exhaustive Search Sort}
Input: $B = ((k_0, v_0), ..., (k_{n-1}, v_{n-1}))$
for each perm $\pi$ :


if $k_{\pi(0)} \leq k_{\pi(1) \leq ... \leq k_{\pi{n-1}}}$:


	return $((k_{\pi(0)}, v_{\pi(0)}), ...., (k_{\pi(n-1)},v_{\pi(n-1)})$
\subsection{Insertion Sort}
Input: $B = ((k_0, v_0), ..., (k_{n-1}, v_{n-1}))$

for each $i=0,1,...,n-1$ :

Insert $B(i)$ at correct place in $(B(0), ... , B(i-1))$

return  $B$

\subsection{Merge Sort}
Input: $B = ((k_0, v_0), ..., (k_{n-1}, v_{n-1}))$

if $n \leq 1$, return $B$ 

else if $n=2$ and  $k_0 \leq k_1$, return $B$ 

else if $n=2$ and $k_0 > k_1$, return $((k_1,v_1),(k_0,v_0))$

else:
$i = \ceil{n/2}$ 
$B_i = \text{MergeSort}((k_0,v_0), ... , (k_{i-1},v_{i-1}))$
$B_2 = ... ((k_i, v_i,... (k_{n-1},v_{n-1}))$ 
return $\text{Merge}(B_1,B_2)$
\subsection{Computational Problem}
 \begin{definition}
	 [Computational Problem]
	 A \textit{computational problem} is a triple $\Pi = (\mathcal{I}, \mathcal{O}, f)$, where:
	 \begin{itemize}
		 \item $\mathcal{I}$ is the set of inputs/instances (typically infinity)
		 \item $\mathcal{O}$ is the set of outputs
		\item $f(x)$ is the set of solutions
		\item $\forall x \in \mathcal{I}(f(x) \subseteq \mathcal{O})$
			\begin{itemize}
				\item Some inputs have multiple correct outputs, and thus $f(x) \subseteq \mathcal{O}$ rather than just being an element. Example: If same key, different values. Then there are multiple answers to sorting.
			\end{itemize}
	 \end{itemize}
\end{definition}
\begin{definition}
	[Algorithm]
	An \textit{algorithm} $A$ solves $\Pi$ if:
	\begin{itemize}
		\item $\forall x \in \I$ st $f(x) \neq \emptyset$ then $A(x) \in f(x)$ 
		\item $\forall x \in \I$ if $f(x)= \emptyset$, $A(x) = \perp$
	\end{itemize}
\end{definition}
\begin{remark}
	The second part must be added because there is a possibility that there is no solution and the algorithm must return something.
\end{remark}
\section{Measuring Efficiency}
\begin{definition}
	[Run Time]
	For an algorithm $A$, given a function $\text{size}(x)$ the runtime is:
	\[
		T(n) = \max(x : \text{size}(x) \leq n)
	\]
	where $T: \nats \to \reals^+$.
\end{definition}
\begin{itemize}
	\item $n$ : Length of input, constraint. Consider all input, $x$, with size less than or equal to $n$.
	\item $T(n)$: Maximize \# of basic operations performing $A$ on $x$
	\item $\text{size}(x)$: \# of (key, value) pairs 
	\item Basic Operations: Arithematic operations, manipulating pointers, executing lines of code, assigning variables
	\item Non-basic Operations: Sorting
	\item We take the $\max$ of $x$ because we are interested in the worst case scenario.
	\item  $\text{size}(x) \leq n$ to make sure that $T(n)$ is increasing and valid on real inputs
\end{itemize}
\begin{notation}
	[Big-Oh Notation]	
	Let $f,g: \nats \to \reals^+$. We say:
	\begin{itemize}
		\item $f = O(g) :  \exists c > 0, \forall n \geq k \in \nats \mid f(n) \leq cg(n) $
		\item  $f = \Omega(g) : \exists c > 0, \forall n \geq k \in \nats \mid f(n) \geq cg(n) \leftrightarrow g=O(f)$
		\item $f = \Theta(g)$ :  $f=O(g) \land f=\Omega(g)$
		\item $f = o(g) : \displaystyle\lim_{n \to \infty} \frac{f(n)}{g(n)} = 0$
		\item $f = \omega(g) : \displaystyle\lim_{n \to \infty} \frac{f(n)}{g(n)} = \infty \leftrightarrow g = o(f)$
	\end{itemize}
\end{notation}
\begin{example}
	Find the running time of ESS.
\end{example}
\begin{soln}
	There are $n!$ permutations of input of size $n$. For each permutation, execute $n-1$ steps. Therefore, $T_{ESS}(n) = \Theta(n!(n-1)) = O(n!(n-1)) = \Omega(n!(n-1)) = \Theta(n!n).$
\end{soln}
\begin{example}
	Find the running time of Insertion Sort.
\end{example}
\begin{soln}
	$T_{\text{ins sort}}(n) = \displaystyle O\paren*{\sum_{i=0}^n i} = O(n^2) = \Omega(n^2) = \Theta(n^2)$.
\end{soln}
\begin{example}
	Find the running time of Merge Sort.
\end{example}
\begin{soln}
	$T_{\text{merge sort}}(n) \displaystyle \leq T_{\text{merge sort}}\paren*{\ceil*{\frac{n}{2}}} + T_{\text{merge sort}}\paren*{\floor{\frac{n}{2}}} + \Theta(n) = O(n\log n)$
\end{soln}
\end{document}
